\documentclass[a4paper,12pt]{article}
% Standard-Dokument mit
% Papierformat A4
% Schrift 12-Punkt
\usepackage{tabularx} % Paket f�r Tabellen
\usepackage[left=3.5cm,right=2.5cm,top=2cm,bottom=2cm]{geometry} % Seiteneinrichtung
\usepackage{textcomp,latexsym} % zusaetzliche Symbole
\usepackage{color} % f�r farbige schrift
\usepackage[ansinew]{inputenc} %universelle Zeichensatzkodierung Win/MAC
\usepackage[ngerman]{babel} % neue deutsche Rechtschreibung
\usepackage{graphicx}  %Zum Einf�gen von Grafiken
\usepackage{amsmath, amssymb} % F�r mathematische Formeln
\usepackage{multirow}
\usepackage[table,xcdraw]{xcolor}
\usepackage{adjustbox,lipsum}
\setlength{\parindent}{0mm} %Zeilen einr�cken nach einem absatz wird unterdr�ckt
\setlength{\parskip}{1ex plus0.5ex minus0.2ex} % Absatzabstand etwas groesser
\setlength{\itemsep}{0ex plus0.2ex} % Abstand zweier Listenelemente kleiner
\renewcommand{\baselinestretch}{1.5}\normalsize % Zeilenabstand 1.5
\renewcommand{\theequation}{\arabic{equation}}
\frenchspacing

%-------------------------------------------------------------------------%
% Dokumentbeginn
\begin{document}
\setlength{\parindent}{0em}  %verhindert Einrücken nach \\+Leerzeile

%-------------------------------------------------------------------------%
% Titelblatt und Inhaltsverzeichnis

\begin{titlepage}

\title{
\vspace*{3cm}
\begin{Huge}TITEL\end{Huge}\\
\begin{Large}Fallstudien I (WS 2018/19)\end{Large}\\
\begin{large}bei Prof. Dr. Carsten Jentsch\end{large}\\
\vspace*{10cm}}

\author{
\begin{large}Ortrud Wartlick\end{large}\\
\begin{large}Gruppe 7\end{large}\\
\begin{large}mit Kai-Robin Lange, Layla Nyrabia und Jannik Rubertus\end{large}}

\date{
\vspace*{1cm}
\today}

\end{titlepage}
\maketitle
\thispagestyle{empty}

\newpage
\pagestyle{empty}
\pagenumbering{roman}
\tableofcontents 
\newpage
\pagestyle{plain}
\pagenumbering{arabic}


%-------------------------------------------------------------------------%
% Hier beginnt der Text

\section{Einleitung}
\begin{itemize}
\item Motivation, kurze Beschreibung von Inhalt und Ziel des Projekts
\item Kurze Erläuterung der Vorgehensweise zur Problemlösung
\item Evtl. kurze Darstellung der zentralen Ergebnisse
\item Überblick über die einzelnen Kapitel
\end{itemize}

\section{Problemstellung}

\begin{itemize}
\item Beschreibung des Datenmaterials
\begin{itemize}
\item Hintergrund und Art der Datenerhebung (geplanter Versuch, Beobachtungsstu- die, Fragebogen usw.)
\item Art und Umfang der Stichprobe (Vollerhebung, Schichtung usw.)
\item Beschreibung aller Variablen (inhaltliche Bedeutung, Einheiten usw.)
\item Diskussion der Datenqualität (fehlende Werte, Messgenauigkeit usw.)
\end{itemize}
\item Beschreibung der Ziele des Projekts (inhaltliche und statistische Zielsetzung)
\end{itemize}


\section{Statistische Methoden}
\begin{itemize}
\item Beschreibung der verwendeten statistischen Verfahren, Modelle usw. und welche
Eigenschaften diese besitzen (hier sind auch mathematische Formeln nötig)
\item Angabe der verwendeten Hilfsmittel (Software, statistische Tafeln usw.)
\item Diskussion der zugrunde liegenden Annahmen, Rechtfertigung der Annahmen bzw.
Darstellung, wie sie überprüft werden können
\end{itemize}


\subsection{Programmcode}
Die R-Befehle zur Erzeugung von Abbildung \ref{dichte} lauten

\begin{verbatim}
> x<-seq(-3,3,0.1)
> plot(x,dnorm(x),type="l",ylab="Dichte der Standardnormalverteilung")
\end{verbatim}

\newpage %Eine neue Seite wird begonnen
%


%%%%%%%%%%%%%%%%%%%%%%%%%%%%%%%%%%%%%%%%%%%%%%%%%%%
% Einlesen von Grafiken im eps-Format (zwei Grafiken hintereinander!)

% \begin{figure} \label{Plot}
% \begin{center}
% \includegraphics[width=14.6cm, height=6.8cm]{Test-Plot.eps}\\
% \includegraphics[width=14.6cm, height=6.8cm]{Test-Plot.eps}
% \caption{\it Residualplot und QQ-Plot}
% \end{center}
% \end{figure}
%
% \newpage
%%%%%%%%%%%%%%%%%%%%%%%%%%%%%%%%%%%%%%%%%%%%%%%%%

\section{Statistische Auswertung}
\begin{itemize}
\item Gegebenenfalls Überprüfung der zu Grunde liegenden Annahmen
\item Ausführliche Darstellung der Ergebnisse, aufbereitet mit Hilfe von Tabellen und
Abbildungen
\item Interpretation der Ergebnisse im Hinblick auf die Problemstellung
\end{itemize}

\subsection{Verweise}
Auf Seite \pageref{dichte} finden wir Abbildung \ref{dichte}, welche die Dichte einer Standardnormalverteilung zeigt.
% mit pageref wird die Seitenzahl aufgerufen, auf der wir die Abbildung finden.

\subsection{Mathematische Formeln}

Das arithmetische Mittel ist durch $\bar{x}=\frac{1}{n} \sum_{i=1}^n x_i$ gegeben.\\
Die Stichprobenvarianz berechnet sich durch 
\begin{align}
\label{Varianz} 
s^2(x) =\frac{1}{n-1}\sum_{i=1}^n (x_i - \bar{x})^2
\end{align} 
der mittleren quadratischen
Abweichung vom arithmetschen Mittel.\\

Seien $X_1,\ldots,X_n$
Zufallsvariablen mit Erwartungswert $\mu$ und $\sigma^2$, so hat auch $\bar{X}$
Erwartungswert $\mu$. Beweis:
\begin{align*}
E(\bar{X}) &= E \left( \frac{1}{n} \sum_{i=1}^n X_i \right) \\
    &= \frac{1}{n} E \left( \sum_{i=1}^n X_i \right)  \\
    &= \frac{1}{n} \sum_{i=1}^n E(X_i)  \\
    &= \frac{1}{n} \sum_{i=1}^n \mu     \\
    &= \frac{n}{n} \mu  =\mu
\end{align*}

Auch die Stichprobenvarianz gegeben in (\ref{Varianz}) erf�llt
\[
E(s^2(X))=\sigma^2.
\]

\subsection{Abbildungen}


\begin{figure}[!ht] 
\center{\includegraphics[width=\textwidth]{figures/Abbildung1.png} }  
\caption{Kursverlauf von Bitcoin und MSCI ACWI.} 
\label{Abb1} 
\end{figure}


\begin{figure}[!ht] 
\center{\includegraphics[width=\textwidth]{figures/Abbildung2.png} }  
\caption{Zeitreihenanalyse und saisonale quartalsweise Effekte auf Bitcoin-Renditen.} 
\label{Abb2} 
\end{figure}


\begin{figure}[!ht] 
\flushright{\includegraphics[height=0.9\textheight]{figures/Abbildung3.png} }  
\caption{Quartalsweise Renditen, Volatilitaet und Sharpe Ratio von Bitcoin.} 
\label{Abb3} 
\end{figure}


Mit LaTeX kann man auch h�bsche Tabellen erstellen. So eine ist in Tabelle \ref{Beispieltab} gegeben:

% Please add the following required packages to your document preamble:
% \usepackage[table,xcdraw]{xcolor}
% If you use beamer only pass "xcolor=table" option, i.e. \documentclass[xcolor=table]{beamer}
\begin{table}[!ht]
\adjustbox{max width=\textwidth}{
\begin{tabular}{l|r|r|r|r|r|r|}
\textbf{}           & \multicolumn{2}{c|}{\textbf{Rendite}}                                                           & \multicolumn{2}{c|}{\textbf{Volatilitaet}}                                           & \multicolumn{2}{c|}{\textbf{Sharpe Ratio}}                                                      \\ \cline{2-7} 
\textbf{}           & \multicolumn{1}{c|}{Bitcoin}            & \multicolumn{1}{c|}{{\color[HTML]{C0C0C0} MSCI ACWI}} & \multicolumn{1}{c|}{Bitcoin} & \multicolumn{1}{c|}{{\color[HTML]{C0C0C0} MSCI ACWI}} & \multicolumn{1}{c|}{Bitcoin}            & \multicolumn{1}{c|}{{\color[HTML]{C0C0C0} MSCI ACWI}} \\ \hline
\textbf{1. Quartal} & \cellcolor[HTML]{FFCCC9}-0.2378 $\pm$ 0.3045 & {\color[HTML]{C0C0C0} 0.0061 $\pm$ 0.0182}                 & 0.0511 $\pm$ 0.0126               & {\color[HTML]{C0C0C0} 0.0070 $\pm$ 0.0023}                 & \cellcolor[HTML]{FFCCC9}-0.0217 $\pm$ 0.0910 & {\color[HTML]{C0C0C0} 0.0184 $\pm$ 0.0376}                 \\ \hline
\textbf{2. Quartal} & 0.2386 $\pm$ 0.5198                          & {\color[HTML]{C0C0C0} 0.0026 $\pm$ 0.0082}                 & 0.0362 $\pm$ 0.0085               & {\color[HTML]{C0C0C0} 0.0057 $\pm$ 0.0022}                 & 0.0861 $\pm$ 0.1366                          & {\color[HTML]{C0C0C0} 0.0090 $\pm$ 0.0200}                 \\ \hline
\textbf{3. Quartal} & \cellcolor[HTML]{FFCCC9}-0.0300 $\pm$ 0.3264 & {\color[HTML]{C0C0C0} 0.0377 $\pm$ 0.0152}                 & 0.0283 $\pm$ 0.0086               & {\color[HTML]{C0C0C0} 0.0046 $\pm$ 0.0012}                 & \cellcolor[HTML]{FFCCC9}-0.0072 $\pm$ 0.1061 & {\color[HTML]{C0C0C0} 0.1272 $\pm$ 0.1056}                 \\ \hline
\textbf{4. Quartal} & 0.7023 $\pm$ 0.7398                          & {\color[HTML]{C0C0C0} 0.0324 $\pm$ 0.0316}                 & 0.0337 $\pm$ 0.0129               & {\color[HTML]{C0C0C0} 0.0047 $\pm$ 0.0028}                 & 0.2232 $\pm$ 0.0542                          & {\color[HTML]{C0C0C0} 0.0976 $\pm$ 0.1008}                 \\ \hline
\end{tabular}
}\\
\caption{Mediale quartalsweise Renditen, Volatilitaet und Sharpe-Ratio von Bitcoin und MSCI ACWI.}
\label{Tab1}
\end{table}




\section{Zusammenfassung}
\begin{itemize}
\item Kurze Wiederholung der Fragestellung des Projekts
\item Kurze Darstellung der wichtigsten Ergebnisse
\item Diskussion der Ergebnisse (mögliche Schlussfolgerungen, Warnung vor Fehlinterpre-
tationen usw.)
\item Ausblick (offene Fragen, Hinweis auf mögliche weitere Untersuchungen usw.)
\end{itemize}

% \begin{align*}...\end{align*} f�r mathematische Umgebung. * Unterdr�ckt Numerierung der Formeln
% \\ erzwingt einen Zeilenumbruch
% & bewirkt, dass in die Zeichen danach in den Zeilen b�ndig stehen
% \left und \right passen die gr��e des Zeichens (hier: "(" und ")" ) der gr��e der Zeichen dazwischen an

\section*{Literaturverzeichnis} % ohne Nummer
Schumacher, M. und Schulgen, G. (2008): Methodik klinischer Studien, 3. Auflage,
Springer, Berlin.\\
\\
Byth, K., Cox, D. R. und Forder, P. (2006): Assessing the relationship between symptoms of allergic rhinoconjunctivitis and pollen counts, Australian and New Zealand Journal of Statistics 48(4), 417–428.\\
\\
Bei Web-Seiten muss das Datum der Abfrage mit angegeben werden.\\
\\
Alle im Literaturverzeichnis aufgeführten Arbeiten müssen im Text zitiert werden
und umgekehrt.\\

\appendix
\newpage
\section*{Anhang}
\addcontentsline{toc}{section}{Anhang}
\renewcommand{\thesubsection}{\Alph{subsection}}

\subsection{Erklärung zum Bericht}

\subsection{Tabellen und Grafiken}

% Please add the following required packages to your document preamble:
% \usepackage{multirow}
% \usepackage[table,xcdraw]{xcolor}
% If you use beamer only pass "xcolor=table" option, i.e. \documentclass[xcolor=table]{beamer}
\begin{table}[h]
\adjustbox{max width=\textwidth}{
\begin{tabular}{ll|r|r|r|r|}
\textbf{}                                               &                                  & \textbf{Enddatum}               & \textbf{Endkurs (USD)}              & \textbf{Gewinn (USD)}              & \textbf{Rendite}              \\ \hline
\multicolumn{1}{l|}{}                                   & Bitcoin                          & 07/12/17                        & 17899.70                      & 3633.60                      & 0.2547                        \\ \cline{2-6} 
\multicolumn{1}{l|}{\multirow{-2}{*}{\textbf{Tag}}}     & {\color[HTML]{C0C0C0} MSCI ACWI} & {\color[HTML]{C0C0C0} 22/01/16} & {\color[HTML]{C0C0C0} 367.97} & {\color[HTML]{C0C0C0} 9.52}  & {\color[HTML]{C0C0C0} 0.0265} \\ \hline
\multicolumn{1}{l|}{}                                   & Bitcoin                          & 10/12/17                        & 15455.40                      & 4140.00                      & 0.3658                        \\ \cline{2-6} 
\multicolumn{1}{l|}{\multirow{-2}{*}{\textbf{Woche}}}   & {\color[HTML]{C0C0C0} MSCI ACWI} & {\color[HTML]{C0C0C0} 16/02/18} & {\color[HTML]{C0C0C0} 522.60} & {\color[HTML]{C0C0C0} 21.69} & {\color[HTML]{C0C0C0} 0.0433} \\ \hline
\multicolumn{1}{l|}{}                                   & Bitcoin                          & 31/12/17                        & 14156.40                      & 3957.80                      & 0.3880                        \\ \cline{2-6} 
\multicolumn{1}{l|}{\multirow{-2}{*}{\textbf{Monat}}}   & {\color[HTML]{C0C0C0} MSCI ACWI} & {\color[HTML]{C0C0C0} 30/10/15} & {\color[HTML]{C0C0C0} 411.24} & {\color[HTML]{C0C0C0} 29.59} & {\color[HTML]{C0C0C0} 0.0775} \\ \hline
\multicolumn{1}{l|}{}                                   & Bitcoin                          & 31/12/17                        & 14156.40                      & 9815.35                      & 2.2610                        \\ \cline{2-6} 
\multicolumn{1}{l|}{\multirow{-2}{*}{\textbf{Quartal}}} & {\color[HTML]{C0C0C0} MSCI ACWI} & {\color[HTML]{C0C0C0} 31/03/17} & {\color[HTML]{C0C0C0} 448.86} & {\color[HTML]{C0C0C0} 27.02} & {\color[HTML]{C0C0C0} 0.0640} \\ \hline
\multicolumn{1}{l|}{}                                   & Bitcoin                          & 31/12/17                        & 14156.40                      & 13192.74                     & 13.690                        \\ \cline{2-6} 
\multicolumn{1}{l|}{\multirow{-2}{*}{\textbf{Jahr}}}    & {\color[HTML]{C0C0C0} MSCI ACWI} & {\color[HTML]{C0C0C0} 29/12/17} & {\color[HTML]{C0C0C0} 513.02} & {\color[HTML]{C0C0C0} 91.18} & {\color[HTML]{C0C0C0} 0.2161} \\ \hline
\end{tabular}
}\\
\caption{Maximale Gewinne von Bitcoin und MSCI ACWI.}
\label{AnhangTab1}
\end{table}

% Please add the following required packages to your document preamble:
% \usepackage{multirow}
% \usepackage[table,xcdraw]{xcolor}
% If you use beamer only pass "xcolor=table" option, i.e. \documentclass[xcolor=table]{beamer}
\begin{table}[h]
\adjustbox{max width=\textwidth}{
\begin{tabular}{ll|r|r|r|r|}
\textbf{}                                               & \textbf{}                        & \textbf{Enddatum}               & \textbf{Endkurs (USD)}               & \textbf{Verlust (USD)}              & \textbf{Rendite}               \\ \hline
\multicolumn{1}{l|}{}                                   & Bitcoin                          & 16/01/18                        & 11490.50                      & -2345.60                      & -0.1695                        \\ \cline{2-6} 
\multicolumn{1}{l|}{\multirow{-2}{*}{\textbf{Tag}}}     & {\color[HTML]{C0C0C0} MSCI ACWI} & {\color[HTML]{C0C0C0} 24/06/16} & {\color[HTML]{C0C0C0} 388.26} & {\color[HTML]{C0C0C0} -19.42} & {\color[HTML]{C0C0C0} -0.0476} \\ \hline
\multicolumn{1}{l|}{}                                   & Bitcoin                          & 24/12/17                        & 13925.80                      & -5180.60                      & -0.2711                        \\ \cline{2-6} 
\multicolumn{1}{l|}{\multirow{-2}{*}{\textbf{Woche}}}   & {\color[HTML]{C0C0C0} MSCI ACWI} & {\color[HTML]{C0C0C0} 09/02/18} & {\color[HTML]{C0C0C0} 500.90} & {\color[HTML]{C0C0C0} -30.55} & {\color[HTML]{C0C0C0} -0.0574} \\ \hline
\multicolumn{1}{l|}{}                                   & Bitcoin                          & 31/01/18                        & 10221.10                      & -3891.10                      & -0.2757                        \\ \cline{2-6} 
\multicolumn{1}{l|}{\multirow{-2}{*}{\textbf{Monat}}}   & {\color[HTML]{C0C0C0} MSCI ACWI} & {\color[HTML]{C0C0C0} 31/08/15} & {\color[HTML]{C0C0C0} 396.73} & {\color[HTML]{C0C0C0} -30.04} & {\color[HTML]{C0C0C0} -0.0704} \\ \hline
\multicolumn{1}{l|}{}                                   & Bitcoin                          & 31/03/18                        & 6973.53                       & -7138.67                      & -0.5058                        \\ \cline{2-6} 
\multicolumn{1}{l|}{\multirow{-2}{*}{\textbf{Quartal}}} & {\color[HTML]{C0C0C0} MSCI ACWI} & {\color[HTML]{C0C0C0} 30/09/15} & {\color[HTML]{C0C0C0} 381.65} & {\color[HTML]{C0C0C0} -41.85} & {\color[HTML]{C0C0C0} -0.0988} \\ \hline
\multicolumn{1}{l|}{}                                   & Bitcoin                          & 30/09/18                        & 6625.56                       & -7486.64                      & -0.5305                        \\ \cline{2-6} 
\multicolumn{1}{l|}{\multirow{-2}{*}{\textbf{Jahr}}}    & {\color[HTML]{C0C0C0} MSCI ACWI} & {\color[HTML]{C0C0C0} 31/12/15} & {\color[HTML]{C0C0C0} 399.36} & {\color[HTML]{C0C0C0} -17.75} & {\color[HTML]{C0C0C0} -0.0425} \\ \hline
\end{tabular}
}\\
\caption{Maximale Verluste von Bitcoin und MSCI ACWI.}
\label{AnhangTab2}
\end{table}


% Please add the following required packages to your document preamble:
% \usepackage[table,xcdraw]{xcolor}
% If you use beamer only pass "xcolor=table" option, i.e. \documentclass[xcolor=table]{beamer}
\begin{table}[h]
\adjustbox{max width=\textwidth}{
\begin{tabular}{l|r|r|r|r|r|r|}
                   & \multicolumn{2}{c|}{\textbf{Rendite}}                                                                    & \multicolumn{2}{c|}{\textbf{Volatilitaet}}                                                             & \multicolumn{2}{c|}{\textbf{Sharpe Ratio}}                                                               \\ \cline{2-7} 
                   & \multicolumn{1}{c|}{\textbf{Bitcoin}}   & \multicolumn{1}{c|}{{\color[HTML]{C0C0C0} \textbf{MSCI ACWI}}} & \multicolumn{1}{c|}{\textbf{Bitcoin}} & \multicolumn{1}{c|}{{\color[HTML]{C0C0C0} \textbf{MSCI ACWI}}} & \multicolumn{1}{c|}{\textbf{Bitcoin}}   & \multicolumn{1}{c|}{{\color[HTML]{C0C0C0} \textbf{MSCI ACWI}}} \\ \hline
\textbf{Januar}    & \cellcolor[HTML]{FFCCC9}-0.1438 $\pm$ 0.2236 & \cellcolor[HTML]{FFCCC9}{\color[HTML]{C0C0C0} -0.0163 $\pm$ 0.0638} & 0.0515 $\pm$ 0.0136                        & {\color[HTML]{C0C0C0} 0.0057 $\pm$ 0.0029}                          & \cellcolor[HTML]{FFCCC9}-0.0947 $\pm$ 0.0551 & \cellcolor[HTML]{FFCCC9}{\color[HTML]{C0C0C0} -0.0896 $\pm$ 0.3333} \\ \hline
\textbf{Februar}   & 0.1724 $\pm$ 0.0636                          & {\color[HTML]{C0C0C0} 0.0263 $\pm$ 0.0411}                          & 0.0384 $\pm$ 0.0262                        & {\color[HTML]{C0C0C0} 0.0061 $\pm$ 0.0051}                          & 0.1663 $\pm$ 0.1828                          & {\color[HTML]{C0C0C0} 0.3754 $\pm$ 0.2707}                          \\ \hline
\textbf{Maerz}     & \cellcolor[HTML]{FFCCC9}-0.0917 $\pm$ 0.0773 & {\color[HTML]{C0C0C0} 0.0022 $\pm$ 0.0297}                          & 0.0439 $\pm$ 0.0140                        & {\color[HTML]{C0C0C0} 0.0075 $\pm$ 0.0012}                          & \cellcolor[HTML]{FFCCC9}-0.1077 $\pm$ 0.0873 & {\color[HTML]{C0C0C0} 0.0197 $\pm$ 0.1845}                          \\ \hline
\textbf{April}     & 0.0757 $\pm$ 0.1613                          & {\color[HTML]{C0C0C0} 0.0128 $\pm$ 0.0075}                          & 0.0243 $\pm$ 0.0179                        & {\color[HTML]{C0C0C0} 0.0055 $\pm$ 0.0013}                          & 0.1677 $\pm$ 0.2161                          & {\color[HTML]{C0C0C0} 0.0989 $\pm$ 0.0627}                          \\ \hline
\textbf{Mai}       & 0.0802 $\pm$ 0.3131                          & \cellcolor[HTML]{FFCCC9}{\color[HTML]{C0C0C0} -0.0018 $\pm$ 0.0034} & 0.0293 $\pm$ 0.0122                        & {\color[HTML]{C0C0C0} 0.0048 $\pm$ 0.0015}                          & 0.1080 $\pm$ 0.3007                          & \cellcolor[HTML]{FFCCC9}{\color[HTML]{C0C0C0} -0.0106 $\pm$ 0.0317} \\ \hline
\textbf{Juni}      & 0.0549 $\pm$ 0.2141                          & \cellcolor[HTML]{FFCCC9}{\color[HTML]{C0C0C0} -0.0070 $\pm$ 0.0145} & 0.0339 $\pm$ 0.0096                        & {\color[HTML]{C0C0C0} 0.0057 $\pm$ 0.0026}                          & 0.0658 $\pm$ 0.2282                          & \cellcolor[HTML]{FFCCC9}{\color[HTML]{C0C0C0} -0.0174 $\pm$ 0.0728} \\ \hline
\textbf{Juli}      & 0.0844 $\pm$ 0.1469                          & {\color[HTML]{C0C0C0} 0.0267 $\pm$ 0.0225}                          & 0.0289 $\pm$ 0.0148                        & {\color[HTML]{C0C0C0} 0.0048 $\pm$ 0.0007}                          & 0.0744 $\pm$ 0.1300                          & {\color[HTML]{C0C0C0} 0.2990 $\pm$ 0.1641}                          \\ \hline
\textbf{August}    & \cellcolor[HTML]{FFCCC9}-0.0864 $\pm$ 0.1513 & {\color[HTML]{C0C0C0} 0.0017 $\pm$ 0.0062}                          & 0.0285 $\pm$ 0.0063                        & {\color[HTML]{C0C0C0} 0.0043 $\pm$ 0.0014}                          & \cellcolor[HTML]{FFCCC9}-0.1103 $\pm$ 0.1438 & {\color[HTML]{C0C0C0} 0.0194 $\pm$ 0.0458}                          \\ \hline
\textbf{September} & \cellcolor[HTML]{FFCCC9}-0.0377 $\pm$ 0.0759 & {\color[HTML]{C0C0C0} 0.0026 $\pm$ 0.0224}                          & 0.0261 $\pm$ 0.0128                        & {\color[HTML]{C0C0C0} 0.0042 $\pm$ 0.0014}                          & \cellcolor[HTML]{FFCCC9}-0.0132 $\pm$ 0.0810 & {\color[HTML]{C0C0C0} 0.0304 $\pm$ 0.2574}                          \\ \hline
\textbf{Oktober}   & 0.3312 $\pm$ 0.2698                          & {\color[HTML]{C0C0C0} 0.0131 $\pm$ 0.0280}                          & 0.0322 $\pm$ 0.0191                        & {\color[HTML]{C0C0C0} 0.0056 $\pm$ 0.0032}                          & 0.3330 $\pm$ 0.0699                          & {\color[HTML]{C0C0C0} 0.2383 $\pm$ 0.3251}                          \\ \hline
\textbf{November}  & 0.1571 $\pm$ 0.0998                          & {\color[HTML]{C0C0C0} 0.0152 $\pm$ 0.0038}                          & 0.0420 $\pm$ 0.0072                        & {\color[HTML]{C0C0C0} 0.0041 $\pm$ 0.0014}                          & 0.1249 $\pm$ 0.0342                          & {\color[HTML]{C0C0C0} 0.1858 $\pm$ 0.0988}                          \\ \hline
\textbf{Dezember}  & 0.1409 $\pm$ 0.3665                          & {\color[HTML]{C0C0C0} 0.0150 $\pm$ 0.0080}                          & 0.0330 $\pm$ 0.0208                        & {\color[HTML]{C0C0C0} 0.0053 $\pm$ 0.0031}                          & 0.1537 $\pm$ 0.3057                          & {\color[HTML]{C0C0C0} 0.1397 $\pm$ 0.1193}                          \\ \hline
\end{tabular}
}\\
\caption{Mediale monatliche Renditen, Volatilitaet und Sharpe-Ratio von Bitcoin und MSCI ACWI.}
\label{AnhangTab3}
\end{table}


\begin{figure}[h] 
\center{\includegraphics[width=\textwidth]{figures/Anhang_Abbildung1.png}  }
\caption{Zeitreihenanalyse und saisonale monatliche Effekte auf Bitcoin-Renditen.}
\label{AnhangAbb1} 
\end{figure}


\begin{figure}[h] 
\flushright{\includegraphics[height=0.95\textheight]{figures/Anhang_Abbildung2.png}  }
\caption{Monatliche Renditen, Volatilitaet und Sharpe Ratio von Bitcoin.}
\label{AnhangAbb2} 
\end{figure}


\end{document}
